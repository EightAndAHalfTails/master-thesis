% !TEX TS-program = pdflatex
% !TEX encoding = UTF-8 Unicode

% This is a simple template for a LaTeX document using the "article" class.
% See "book", "report", "letter" for other types of document.

\documentclass[11pt,a4paper,twoside,openany]{report}

% Nicer fonts
\usepackage[T1]{fontenc}
\usepackage{lmodern}

%%% Examples of Article customizations
% These packages are optional, depending whether you want the features they provide.
% See the LaTeX Companion or other references for full information.

%%% PAGE DIMENSIONS
\usepackage[inner=2cm,outer=4cm]{geometry}

\usepackage{graphicx} % support the \includegraphics command and options

% \usepackage[parfill]{parskip} % Activate to begin paragraphs with an empty line rather than an indent

%%% PACKAGES
\usepackage{booktabs} % for much better looking tables
\usepackage{array} % for better arrays (eg matrices) in maths
\usepackage{paralist} % very flexible & customisable lists (eg. enumerate/itemize, etc.)
\usepackage{verbatim} % adds environment for commenting out blocks of text & for better verbatim
\usepackage{subfig} % make it possible to include more than one captioned figure/table in a single float
% These packages are all incorporated in the memoir class to one degree or another...

%%% HEADERS & FOOTERS
\usepackage{fancyhdr} % This should be set AFTER setting up the page geometry
\pagestyle{fancy} % options: empty , plain , fancy
\renewcommand{\headrulewidth}{0pt} % customise the layout...
\lhead{}\chead{}\rhead{}
\lfoot{}\cfoot{\thepage}\rfoot{}

%%% ToC (table of contents) APPEARANCE
\usepackage[nottoc,notlof,notlot]{tocbibind} % Put the bibliography in the ToC
\usepackage[titles,subfigure]{tocloft} % Alter the style of the Table of Contents
\renewcommand{\cftsecfont}{\rmfamily\mdseries\upshape}
\renewcommand{\cftsecpagefont}{\rmfamily\mdseries\upshape} % No bold!

% Hyperlinks, URLs etc.
\usepackage{hyperref}
\usepackage{url}
\hypersetup{
    colorlinks=true,
    citecolor=black,
    urlcolor=black,
    linkcolor=black,
    pagecolor=black,
    anchorcolor=black
}

% Listings
%\usepackage{chngcntr} % Renumbering done further down because the
                      % counter is only available after the document
                      % begins, for some reason.
\usepackage{listings}
\usepackage[usenames,dvipsnames]{color}

\definecolor{LightGray}{rgb}{0.9,0.9,0.9}

\lstset{basicstyle=\footnotesize\ttfamily}
\lstset{showstringspaces=false}
\lstset{numbers=none}
\lstset{keywordstyle=\color{MidnightBlue}\bfseries}
\lstset{commentstyle=\color{JungleGreen}}
\lstset{identifierstyle=\color{OliveGreen}}
\lstset{stringstyle=\color{Red}}
\lstset{backgroundcolor=\color{LightGray}}
\lstset{breaklines=true}
\lstset{captionpos=b}

%\makeatletter
%\renewcommand\l@lstlisting[2]{\@dottedtocline{1}{1.5em}{2.75em}{#1}{#2}}
%\makeatother

%%% END Article customizations

%%% The "real" document content comes below...

\title{%
{\normalsize\scshape Imperial College London}\\
{\normalsize\scshape Telecom Paristech}\\
\vspace{2em}
\hrule
\vspace{1em}
{\huge\bfseries Hardware-based Buffer Overflow Protection}\\
\vspace{1em}
\hrule
\vspace{1em}
{\small Submitted for part completion of MEng Electronic and Information Engineering}
}
\author{%
Author: Jake Humphrey\\
Supervisor: Guillaume Duc
}
\date{}

\begin{document}

%\counterwithin{lstlisting}{section}

\pagestyle{empty}

\maketitle

\cleardoublepage
\begin{abstract}
Implementing security features in hardware can free the software programmer from worrying about implementing such security, leading in turn to simplified code. This can be especially useful when programming in low-level languages such as C, which provide minimal built-in protection.

The work detailed in this project report modifies the Memory Management Unit (MMU) of an existing processor to implement protection against buffer overflow attacks, then produces some assembly code showing how firmware or an operating system kernel would make use of the protection. Finally, it analyses the effect of the added protection on hardware area and clock frequency.

\end{abstract}

\cleardoublepage
\pagestyle{plain}
\pagenumbering{roman}
\setcounter{page}{1}
\tableofcontents

\newpage
\cleardoublepage
\pagenumbering{arabic}

\chapter{Introduction}
Today computers have access to large amounts of our personal data. It can often have disastrous effects on our personal and professional lives if the data is compromised. The problem with leaving security issues to software developers is that often multiple processes will be running on the same system, and a vulnerability in any one of them can potentially leak sensitive data, regardless of how secure the others are.

It is sometimes preferable to place security measures at the lower levels of execution, i.e. the hardware, kernel or operating system (OS). This restricts system security to a single point-of-failure, and can result in easier, faster, and more secure code.

\section{Motivation}
A \textbf{Buffer Overflow Attack} makes use of out-of-range array accesses to read or write memory locations intended to be inaccessible. They are typically associated with C and C++, which provide no built-in protection against array accesses falling outside the bounds of the array. Using a Buffer Overflow Attack, an attacker could, for example, corrupt the return address inside a function to point to and thus run arbitrary code.

The Heartbleed bug in OpenSSL\cite{heartbleed} was an example of a vulnerability to this type of attack. An attacker was able to send a server a payload text string along with its length and have the server reply with the exact same payload. However, if the length supplied was longer than the actual length of the payload, the returned message would be appended with whatever followed the text string in memory.\footnote{\url{http://xkcd.com/1354/}} This potentially allowed an attacker access to server private keys, user passwords, or really anything stored on the server's memory.

\section{Existing Solutions}
A potential solution would be to use other programming languages or the Standard Template Library container classes in C++, but for those who cannot afford the performance cut, there exist other protection strategies. For example, the use of \textbf{canary values} involves placing a known value after the buffer. Since buffer overflows are often carried out with such C functions as \verb|strcpy|, they affect contiguous memory areas. A write would therefore alter the canary value, and thus the program can infer that a buffer overflow has occurred and can, for example, invalidate the memory area as a countermeasure.

The above solution works for writes, which corrupt the canary value, but would offer no protection against out-of-bounds reads, such as those making use of the Heartbleed vulnerability. Fortunately, the same idea can be extended to protect against both types of attack, though at significant cost, as will be explained.

\section{Project Aims}
One could imagine a system making use of the Memory Management Unit (MMU), whereby a buffer would be placed immediately before a page boundary, and the following page marked as invalid in the MMU. A buffer overflow, whether a read or a write, would then trigger a page fault, which could potentially allow the OS to resolve the situation.

The biggest downside to this method is that it requires an entire virtual memory page be invalidated. On x86 processors, for example, this is 4KiB. It would be nice to have a smaller invalidatable section of memory in the MMU to use for this purpose.

This project investigates a method to allow the invalidation of smaller memory sections within the the MMU, specifically by adding hardware registers to allow for memory invalidation at the cache block level; in x86 the cache block size is 64B. This allows use of the above protection system without excess wasted memory, however it adds extra silicon in the form of the hardware registers. This tradeoff is also examined during the project.

In addition, a small amount of code is written to both test the new hardware and demonstrate how firmware or software might make use of it.

\section{Report Outline}
This report is made up of 4 chapters:

Chapter~\ref{cha:res} summarises the various processors that could be chosen on which to base the project, followed by the choice made and justifications.

Chapter~\ref{cha:imp} gives an account of the work done during the project, starting with setting up the processor to obtain a development environment, and going on to describe the hardware changes made and code written.

Chapter~\ref{cha:results} describes the testing done and obtains figures for the area and frequency impact of the new hardware, followed by an evaluation of what these results imply for the feasability of the new system.
\chapter{Research}
\label{cha:res}

To carry out the project aims, a \textbf{Soft Processor}, a processor which is to some extent configurable, is needed. Specifically we need to be able to directly modify the MMU code. This chapter provides a comparison of the various Soft Processors available.

\section{Comparison of Soft Processors}
The following metrics will be taken into account:
\begin{itemize}
\item Bus Used
\item Architecture, including Instruction Set Architecture (ISA)
\item Cache
\item Memory Management Unit (MMU)
\item Protection/User Levels
\item License
\item Documentation
\item Toolchain
\item Maturity
\end{itemize}

Basic information was taken from the Wikipedia page on soft processors.\cite{wiki-sp} More detailed information unless otherwise cited was taken from the relevant company's website.\cite{or1k-home}\cite{leon-home}\cite{lm32-home}\cite{mb-home}\cite{nios-home}\cite{m1-home}

\subsection{OpenRISC}
\begin{description}[style=nextline]
\item[Developer] OpenCores
\item[Bus Used] OpenRISC uses Wishbone, an open source specification by OpenCores.
\item[Architecture] 32/64-bit MIPS-like.
\item[Cache] 1--64kB for Instructions and Data each.
\item[MMU] Virtual Memory support. Translation Lookaside Buffer (TLB) with page size 8kB and 16--256 entries.
\item[User Levels] User/Supervisor Mode.
\item[License] There are two main open source implementations available. A Verilog implementation, OpenRISC 1200, is released under LGPL. Another Verilog implementation, mor1kx, is available under a weak copy-left licence created specifically for it, the OHDL.
\item[Documentation] There exists a document detailing the architecture specification.\cite{or1k} Documentation for both implementations are also available at their respective Github pages.\cite{or1200}\cite{mor1kx}
\item[Toolchain] A Compiler (based on gcc) and Debugger (based on gdb) are available.\cite{or1k-elf-gcc}
\item[Maturity] The OpenRISC 1000 specification has been under development since 2001. Version 1.0 was released in 2012, and Version 1.1 in 2014. The OpenRISC 1200 implementation is stable but not in active development, and is the widely used version.\cite{openrisc} Mor1kx is still in active development, but is stated to be more sophisticated.
\end{description}

\subsection{LEON2/3/4}
\begin{description}[style=nextline]
\item[Developer] Aeroflex Gaisler
\item[Bus Used] AMBA2
\item[Architecture] 32-bit. ISA based on SPARC-V8
\item[Cache] LEON2: Instruction and Data caches. Set-associative with 1--4 sets and 1--64kB/set. LRR or LRU replacement. LEON3/4: 1--4 sets, 1--256 kB/set. Random, LRR or LRU replacement. LEON4 includes an optional L2 cache. 256-bit internal, 1-4 sets, 16kB--8MB.
\item[MMU] SPARC Reference MMU (SRMMU) with configurable TLB
\item[User Levels] The User and Supervisor modes of SPARC.
\item[License] Gaisler state on their website that LEON2 is available from Atmel (AT697 and AT7913)\cite{gaisler}, but the VHDL source code appears to be on Github.\cite{leon2} LEON3 is available under GPL as part of the GRLIB library.\cite{grlib} LEON4 is only available under a commercial license from Gaisler.
\item[Documentation] LEON3\cite{leon3} and LEON4\cite{leon4} have documentation.
\item[Toolchain] Being SPARC V8 conformant, compilers and kernels for SPARC V8 can be used with LEON processors. Cross compiler BCC.\cite{bcc} gdb can be used for debugging.
\item[Maturity] No information found.
\end{description}

\subsection{Lm32}
\begin{description}[style=nextline]
\item[Developer] Lattice
\item[Bus Used] Wishbone
\item[Architecture] 32-bit. ISA documentation is available.\cite{lm32-isa}
\item[Cache] Configurable between no cache, 8kB I-cache, and 8kB I-cache and 8kB D-cache.
\item[MMU] None.
\item[User Levels] No separate modes.
\item[License] LatticeMico32 is licensed under a free (IP) core license.
\item[Documentation] Documentation is available from the LatticeSemi website, but registration is required.\cite{lm32-doc}
\item[Toolchain] GCC supports the LM32 since version 4.5.0, and Binutils since 2.19. gdb and Newlib are also supported.
\item[Maturity] Introduced in 2006.
\end{description}

\subsection{Microblaze}
\begin{description}[style=nextline]
\item[Developer] Xilinx
\item[Bus Used] Customisable between CoreConnect PLB, OPB, FSL, LMB, AXI4.
\item[Architecture] In terms of its instruction set architecture, MicroBlaze is very similar to the RISC-based DLX architecture, which in turn is based upon the MIPS architecture.
\item[Cache] 2kB--64kB direct mapped write-through or write-back.
\item[MMU] Optional. 4GB virtual memory, supports page sizes of 1kB, 4kB, 16kB, 64kB, 256kB, 1MB, 4MB, and 16MB. Page-replacement strategy controlled by software.
\item[User Levels] User and privileged modes.
\item[License] Proprietary. Available from Xilinx along with the EDK (Embedded Development Kit). A smaller, less functional version with limited target devices is available for free, but the full version costs \$3000--\$6000, depending on the license type. The license also restricts the use of Microblaze to Xilinx FPGAs. LGPL clones also exist in Verilog, VHDL, or MyHDL.\cite{mb-clone}
\item[Documentation] Rather comprehensive reference manual available.\cite{mb-doc}
\item[Toolchain] Xilinx provides the Vivado design suite, including an IDE, SDK, Simulator, Analyser, and Synthesiser. Support for Microblaze also exists in gcc as of version 4.6.
\item[Maturity] Currently in version 8.50b, under development since 2002.
\end{description}

\subsection{Nios II}
\begin{description}[style=nextline]
\item[Developer] Altera
\item[Bus Used] Altera's own Avalon bus.
\item[Architecture] 32-bit RISC architecture, with support for up to 256 custom instructions. 3 different configurations are available: fast (f), standard (s), and economy (e).
\item[Cache] Nios II/f has separate instruction and data caches of 512B to 64kB, while the Nios II/s has only an instruction cache, and the Nios II/e has none.
\item[MMU] There is an optional MMU and MPU\footnote{Memory Protection Unit} for the Nios II/f. Default is 16-way set-associative cache, the default size depends on the target FPGA but is either 128 or 256 entries or 4 or 8kB. Replacement strategy is set by the software.
\item[User Levels] The Nios II supports user and supervisor modes only when the design includes an MMU or MPU.
\item[License] The License for the Nios II IP core is around \$500. Currently the Altera website has a redirect loop and I am unable to find information on toolchain licenses.
\item[Documentation] Both Hardware and Software development documentation are available.\cite{nios-doc}
\item[Toolchain] Qsys and Quartus for hardware, Altera's Embedded Design Suite for software, based on Eclipse and the GNU toolchain.
\item[Maturity] In development since 2004, currently in version 13.1, released in 2013.
\end{description}

\subsection{Cortex-M1}
\begin{description}[style=nextline]
\item[Developer] ARM
\item[Bus Used] ARM Corelink Interconnect
\item[Architecture] Three-stage 32-bit RISC, ARMv6M ISA.
\item[Cache] No cache, TCM only, up to 1MB each.
\item[MMU] None.
\item[User Levels] None.
\item[License] Proprietary, must contact sales team for cost.
\item[Documentation] No reference manual specifically for the M1. See the ``resources'' tab on the Cortex M1 home page for more information.\cite{m1-home}
\item[Toolchain] An extension of Altera's Quartus II. Supported by the ARM microprocessor development kit for software development.
\item[Maturity] No information found.
\end{description}

\section{Choice and Justification}
As well as having to contain an MMU, another requirement for the project is that the processor have open code for it. Just having a customisable size is not enough, since the aims include adding entirely new registers and logic to the MMU.

These restrictions eliminate all the above save OpenRISC and the LEON2/3 processors. The OpenRISC was chosen for the project because its architecture is closer to the MIPS and ARM architectures commonly in use today, while the SPARC architecture of the LEON processors would be more difficult to learn to program for.
\chapter{Implementation}
\label{cha:imp}

This chapter gives an account of the work done during the project. As described in Chapter~\ref{cha:res}, the OpenRISC (\textbf{or1k}) was chosen as the target processor architecture. To allow for repeated instatiations and modifications of this hardware design, a Field Programmable Gate Array (FPGA) development board was used. Initially the Altera SoCKit\cite{sockit} was chosen as the target device. However, even after extensive modifications, a working build was not obtained. The details of the attempt will nonetheless be detailed both as a record of work done and to advise future attempts.

Following this attempt, an Altera DE2 board\cite{de2} was used. This target device proved successful, and the hardware modifications and code written that are detailed in the following sections were run on this board (although in theory, the code is device-agnostic).

\section{SoCKit}
\label{sec:sockit}
The initial project aim was to use the Altera SoCKit to instantiate and test the hardware.  This section details the (ultimately unsuccessful) attempt to get it working.

\subsection{Building}
The first step was to build the OpenRISC processor and program it onto the board. For this a tool called \textbf{FuseSoC}\cite{fusesoc} (formerly \textbf{OrpSoCv3}) was used. FuseSoC's readme describes it as a ``package manager and a set of build tools for HDL [Hardware Description Language] code.'' Using this, it is simple\footnote{In theory.} to build \textbf{mor1kx} (the OpenRISC implementation in question) for the SoCKit with the command \texttt{fusesoc build sockit}. This command invokes the relevant Quartus executables to build a netlist to be programmed.

However, the build did not work at first. A lot of searching revealed the problem to be a bug in FuseSoC's libraries. As shown below, the port names between two files did not agree. Patching this error allowed the build to continue.\footnote{Unfortunately for me, by the time I thought to submit a bug report or a pull request upstream, the bug had alread been found and patched by the developers.}

Figure~\ref{fig:wb_dec} gives the declaration of the \verb|avalon_to_wb_bridge| module. If this is compared to an instantiation of that module, such as the one given in Figure~\ref{fig:wb_inst}, it can be seen that several port names are incorrect. This is likely due to the former file being changed without proper care being taken to update port names throughout the code.

Once the build succeeded, it could be programmed onto the board using the command \texttt{fusesoc pgm sockit}.

\begin{figure}[t]
  \centering
  \begin{lstlisting}[language=Verilog]
module avalon_to_wb_bridge \#(
	parameter DW = 32,	// Data width
	parameter AW = 32	// Address width
)(
	input 		  wb_clk_i,
	input 		  wb_rst_i,
	// Avalon Slave input
	input [AW-1:0] 	  s_av_address_i,
	input [DW/8-1:0]  s_av_byteenable_i,
	input 		  s_av_read_i,
	output [DW-1:0]   s_av_readdata_o,
	input [7:0] 	  s_av_burstcount_i,
	input 		  s_av_write_i,
	input [DW-1:0] 	  s_av_writedata_i,
	output 		  s_av_waitrequest_o,
	output 		  s_av_readdatavalid_o,
	// Wishbone Master Output
	output [AW-1:0]   wbm_adr_o,
	output [DW-1:0]   wbm_dat_o,
	output [DW/8-1:0] wbm_sel_o,
	output 		  wbm_we_o,
	output 		  wbm_cyc_o,
	output 		  wbm_stb_o,
	output [2:0] 	  wbm_cti_o,
	output [1:0] 	  wbm_bte_o,
	input [DW-1:0] 	  wbm_dat_i,
	input 		  wbm_ack_i,
	input 		  wbm_err_i,
	input 		  wbm_rty_i
);
  \end{lstlisting}
  \caption{The declaration of the avalon\_to\_wb\_bridge module\cite{wb_dec}}
  \label{fig:wb_dec}
\end{figure}

\begin{figure}[t]
  \centering
  \begin{lstlisting}[language=Verilog]
wb_to_avalon_bridge #(
	.DW			(32),
	.AW			(32),
	.BURST_SUPPORT		(1)
) hps_ddr3_wb2avl_bridge (
	.wb_clk_i		(wb_clk),
	.wb_rst_i		(wb_rst),
	// Wishbone Slave Input
	.wb_adr_i		(wb_m2s_hps_ddr3_adr),
	.wb_dat_i		(wb_m2s_hps_ddr3_dat),
	.wb_sel_i		(wb_m2s_hps_ddr3_sel),
	.wb_we_i		(wb_m2s_hps_ddr3_we),
	.wb_cyc_i		(wb_m2s_hps_ddr3_cyc),
	.wb_stb_i		(wb_m2s_hps_ddr3_stb),
	.wb_cti_i		(wb_m2s_hps_ddr3_cti),
	.wb_bte_i		(wb_m2s_hps_ddr3_bte),
	.wb_dat_o		(wb_s2m_hps_ddr3_dat),
	.wb_ack_o		(wb_s2m_hps_ddr3_ack),
	.wb_err_o		(wb_s2m_hps_ddr3_err),
	.wb_rty_o		(wb_s2m_hps_ddr3_rty),
	// Avalon Master Output
	.m_av_address_o		(avm_hps_ddr3_address),
	.m_av_byteenable_o	(hps_0_f2h_sdram0_data_byteenable),
	.m_av_read_o		(hps_0_f2h_sdram0_data_read),
	.m_av_readdata_i	(hps_0_f2h_sdram0_data_readdata),
	.m_av_burstcount_o	(hps_0_f2h_sdram0_data_burstcount),
	.m_av_write_o		(hps_0_f2h_sdram0_data_write),
	.m_av_writedata_o	(hps_0_f2h_sdram0_data_writedata),
	.m_av_waitrequest_i	(hps_0_f2h_sdram0_data_waitrequest),
	.m_av_readdatavalid_i	(hps_0_f2h_sdram0_data_readdatavalid)
);
  \end{lstlisting}
  \caption{An example attempt to instantiate the avalon\_to\_wb\_bridge module\cite{wb_inst}}
  \label{fig:wb_inst}
\end{figure}

\subsection{Running}
After the processor was built and the board programmed, the next step was to load an object file onto the processor. For this a linux image was used, compiled using the \texttt{or1k-elf-gcc} toolchain as detailed on the OpenCores website\cite{or1k-linux}. The resulting object file was simulated under the toolchain's simulator and found to run as expected.

To debug the running processor, including loading object files, the \textbf{Open On-chip Debugger} (\textbf{OpenOCD}) was used as detailed, once again, on the OpenCores website\cite{or1k-openocd}.

However, running OpenOCD for the SoCKit gave an error regarding mismatched clock speeds. Some searching revealed a discussion board\cite{openocd-sockit} where the error was mentioned, which pointed to a patch for the issue\cite{openocd-fix}.

Compiling a version of OpenOCD with this patch applied seemed to get around the issue, but unfortunately it would still not run.\footnote{At time of writing, I've long since forgotten the exact issue, if indeed I ever worked out what it was. Let this be a lesson to write this kind of thing down somewhere.} After a meeting between the Author and Supervisor, it was decided to scrap the SoCKit route and work with the DE2 board, which had been used successfully in a previous project.

\section{DE2}
This section details the work done in setting up a development environment for the DE2 board. These steps were largely similar to those taken in Section~\ref{sec:sockit}.

\subsection{Building}
Once again, Fusesoc was used to build the processor. Appropriate configuration files were provided from a previous project by the project supervisor.

Support for the Cyclone II FPGA, as is used in the DE2, has been discontinued in later versions of Quartus, so version 13.0 needed to be installed for the build to work.

After moving the DE2 files to the appropriate Fusesoc directory, \texttt{fusesoc build de2} and \texttt{fusesoc pgm de2} worked as expected.

\subsection{Running}
Running OpenOCD proved far more successful this time, though in place of a Linux image, a Hello World program provided from the previous DE2 project was used.

\section{Hardware}
\label{sec:har}
With a development environment finally obtained, the actual aim of the project could be implemented: the addition of hardware registers and logic to allow for invalidation of virtual memory of cache block size within the Data MMU (DMMU).

The first step was to add the new registers. The processor implementation being used was \textbf{mor1kx}\cite{mor1kx}. This implementation was configured to have a cache block size of 32B, and a DMMU page size of 8kB. This results in $\frac{8kB}{32B} = 256$ cache lines (and therefore invalidation bits required) per page.

In addition, the DMMU has 1 way and 64 sets. With 64 potential pages in the DMMU, this results in a grand total of $64 \times 256 = 16384$ extra bits. Since the mor1kx word size is 32 bits, this could be achieved with the addition of $\frac{16384}{32} = 512$ extra 32-bit Special Purpose Registers (SPRs).

SPRs are a feature of the or1k architecture which are generally used to hold configurations and data specific to a unit within the processor. For example, the Translation Lookaside Buffer (TLB) within the DMMU is implemented as a series of ``match'' and ``translate'' SPRs. SPRs are accessible to software via the supervisor-mode instructions \texttt{mtspr} and \texttt{mfspr} (move to\slash from SPR).

Each unit has a ``group'' of 2048 SPR addresses allocated to it, and fortunately the last 512 addresses in the DMMU group were unused, allowing their allocation for the purposes of this project.

Figure~\ref{fig:regs_inst} shows the instantiation of the new registers. Alhough the address width was given in terms of the set width\footnote{6 bits for 64 sets, +8 for 256 bits per page, -5 for the 32 bits in each SPR}, increasing the number of sets or ways in the DMMU now would cause the addresses of the new registers to overflow their allocated address space. Therefore, the DMMU size should henceforth be considered unconfigurable.

\begin{figure}[t]
  \centering
  \begin{lstlisting}[language=Verilog]
generate
   // DTLB protection registers
   mor1kx_simple_dpram_sclk
     #(
       .ADDR_WIDTH(OPTION_DMMU_SET_WIDTH+3),
       .DATA_WIDTH(OPTION_OPERAND_WIDTH)
       )
   dtlb_protection_regs
     (
      // Outputs
      .dout				(dtlb_protect_dout),
      // Inputs
      .clk				(clk),
      .raddr				(dtlb_protect_addr),
      .re				(1'b1),
      .waddr				(dtlb_protect_addr),
      .we				(dtlb_protect_we),
      .din				(dtlb_protect_din)
      );
endgenerate
  \end{lstlisting}
  \caption{The code instantiating the new registers in the DMMU}
  \label{fig:regs_inst}
\end{figure}

The next step was to add logic allowing the registers to be accessed. There are two ways this might occur:
\begin{itemize}
	\item The \texttt{mtspr} and \texttt{mfspr} instructions allow the CPU to write to or read from the registers directly.
	\item Any address translation should first check to see if the appropriate invalidation bit is set, and if so trigger a page fault.
\end{itemize}

\begin{figure}[t]
  \centering
  \begin{lstlisting}[language=Verilog]
   always @(*) begin
      // Snip block initialisations
      dtlb_protect_we = dtlb_protect_spr_cs & spr_bus_we_i;
      // Snip other write-enable signals
   end

   assign dtlb_match_spr_cs = spr_bus_stb_i
                            & (spr_bus_addr_i[15:11] == 5'd1)
                            & ^spr_bus_addr_i[10:9] & !spr_bus_addr_i[7];
   assign dtlb_trans_spr_cs = spr_bus_stb_i
                            & (spr_bus_addr_i[15:11] == 5'd1)
                            & ^spr_bus_addr_i[10:9] & spr_bus_addr_i[7];
   assign dtlb_protect_spr_cs = spr_bus_stb_i
                              & (spr_bus_addr_i[15:11] == 5'd1)
                              & &spr_bus_addr_i[10:9];

   assign dtlb_match_addr = dtlb_match_spr_cs ?
			    spr_bus_addr_i[OPTION_DMMU_SET_WIDTH-1:0] :
			    virt_addr_i[13+(OPTION_DMMU_SET_WIDTH-1):13];
   assign dtlb_trans_addr = dtlb_trans_spr_cs ?
			    spr_bus_addr_i[OPTION_DMMU_SET_WIDTH-1:0] :
			    virt_addr_i[13+(OPTION_DMMU_SET_WIDTH-1):13];
   assign dtlb_protect_addr = dtlb_protect_spr_cs ?
			      spr_bus_addr_i[OPTION_DMMU_SET_WIDTH+3-1:0] :
			      virt_addr_i[18:10];

   assign dtlb_match_din = dtlb_match_reload_we ? dtlb_match_reload_din :
			   spr_bus_dat_i;
   assign dtlb_trans_din = dtlb_trans_reload_we ? dtlb_trans_reload_din :
			   spr_bus_dat_i;
   assign dtlb_protect_din = spr_bus_dat_i;
   
   // Snip area translate buffer signals
   
   assign spr_bus_dat_o =
         dtlb_protect_spr_cs_r ? dtlb_protect_dout :
         dtlb_match_spr_cs_r ? dtlb_match_dout[spr_way_idx_r] :
         dtlb_trans_spr_cs_r ? dtlb_trans_dout[spr_way_idx_r] :
         dmmucr_spr_cs_r ? dmmucr : 0;
  \end{lstlisting}
  \caption{New logic signals to present the new registers to the SPR bus.}
  \label{fig:spr_logic}
\end{figure}

The \texttt{mtspr} and \texttt{mfspr} instructions take as argument a 16-bit SPR address; the first 5 bits encode the group number, and the last 11 bits the register number within that group. The DMMU SPRs belong to group 1, and we are looking for the last 512 registers\footnote{i.e. register numbers 1536 to 2047}. Therefore, we are looking for an address of the format \texttt{0b0000 111x xxxx xxxx}.

Within the DMMU, all signals relating to these instructions are provided via the SPR bus, and it is via this bus that pertinent data must be sent back to the other sections of the CPU.

Figure~\ref{fig:spr_logic} shows the preexisting code for the match and translate registers, alongside the new code added for the new registers. The \verb|_spr_cs| signals indicate whether an incoming SPR address falls into one of the named SPRs.\footnote{i.e. \texttt{dtlb\_match\_spr\_cs} is asserted if the SPR address points to a match register in the DTLB.} If this signal is asserted, then the register address within the RAM block is taken from the lower 9 bits of the SPR address.

Since the software is going to be in charge of invalidating memory, the registers will only ever be written by an \texttt{mtspr} instruction. Therefore the RAM's input data bus reads directly from the data-in field of the SPR bus, and the write enable signal is asserted on a \texttt{mtspr} instruction.

Finally, the data from the RAM is returned to the SPR bus. The \verb|_spr_cs_r| signals are the \verb|_spr_cs| delayed by one cycle to align with the read cycle. When these are asserted, the appropriate RAM block is selected to read the output from.

Next comes the logic to read the appropriate bit during an address translation. When a virtual address comes in, the logic needs to extract the word and bit address to return the correct bit in the RAM. The virtual address should be decoded as follows:

The lower 5 bits \texttt{[4:0]} refer to the word inside the cache block and should be ignored. Bits \texttt{[12:5]} give the 8-bit index into the 256 bits allocated to each page. Bits \texttt{[18:13]} give the 6-bit set index.

In other words, the 14 bits \texttt{[18:5]} give the bit index within the RAM. Of these, the 9 upper bits \texttt{[18:10]} index one of the 512 registers, while the lower 5 \texttt{[9:5]} index the bit within the register.

Figure~\ref{fig:pgf_logic} shows the lines edited. A new signal was created to indicate when a load or store operation attempts to access invalid memory, then this signal is used alongside the existing access code violations as a condition for the pagefault signal.

\begin{figure}[t]
  \centering
  \begin{lstlisting}[language=Verilog]
   assign dtlb_protect_viol = (op_store_i || op_load_i)
                              && dtlb_protect_dout[virt_addr_i[9:5]];

   assign pagefault_o = ((supervisor_mode_i ?
			!swe & op_store_i || !sre & op_load_i :
			!uwe & op_store_i || !ure & op_load_i)
			|| dtlb_protect_viol) &
			!tlb_reload_busy_o;
  \end{lstlisting}
  \caption{New logic signals to check for invalidated memory during address translation.}
  \label{fig:pgf_logic}
\end{figure}

\section{Software}
On the software side, two programs were needed, one to test the MMU of the base implementation, and one to test the new added features of Section~\ref{sec:har}.

\subsection{Before Hardware Modification}
For the first program, the first step was to write the exception handlers, specifically the reset handler that gets called whenever the processor is reset or turned on, and the Data TLB (DTLB) miss exception handler that occurs whenever a Page Table Entry (PTE) needs to be loaded into the DTLB.

The exception handlers are shown in Figure~\ref{fig:exh}. The reset handler initialises register 0 to a contain a value of 0, as is required by the or1k specification, then sets up registers 1 and 2, the stack and frame pointers respectively. It then jumps to the start of the program.

The DTLB miss exception handler is responsible for loading PTEs into the DTLB. In normal operation this would involve doing a page walk over the Page Table (PT). However, for this project it was not necessary to emulate this behaviour. A PT was not used, and instead the handler simply creates a PTE corresponding to the identity transformation\footnote{i.e. Virtual Address = Physical Address} on the fly.

To set up the relevant entries in the DTLB, the exception handler must fill the match register in the appropriate set with the Virtual Page Number (VPN), and fill the corresponding translate register with the Physical Page Number (PPN). In addition, the match register requires bit 0 to be set to be valid, and the translate register has several protection bits saying whether the corresponding page can be read or written in supervisor or user mode. For the purposes of this test program, these can all be enabled.

In Figure~\ref{fig:exh}, the exception handler first obtains the Effective Address\footnote{For all intents and purposes, this is identical to a virtual address, but the or1k specification distinguishes beween them} (EA) whose translation triggered the exception. The page number of this address is made up of the top 19 bits, so the bottom 13 bits are cleared. The set number is found by taking bits 18 to 13 of the EA. The result is then used as an offset for the \texttt{mtspr} instruction, which takes a register value and ORs it with an immediate to obtain the SPR address.

After the exception handlers, the main program was written. All this had to do was store some data in memory, then activate the DMMU and attempt to retrieve the data. Figure~\ref{fig:mn1} shows the code written to do this. The DMMU is enabled by setting bit 5 of an SPR called the Status Register (SR). The test value is loaded into register 15, and by checking the contents of this register with OpenOCD, it can be confirmed that the program worked as expected.

\begin{figure}[t]
  \centering
  \begin{lstlisting}[language={[x86masm]Assembler}]
	.global _start
	.org 0x100
reset:
	l.andi	r0, r0,	0
	l.movhi	r1, hi(_stack)
	l.ori	r1, r1, lo(_start)
	l.or	r2, r0, r1
	
	l.j	_start
	l.nop

	.org 0x900
dtlbms:
# virt = phys
	l.mfspr	r23, r0, 0x30		# r23 = EA
	l.movhi	r25, 0xffff
	l.ori	r25, r25, 0xe000	# r25 = 0xfffe000
	l.and	r25, r25, r23		# r25 = EA[31:13] = VPN/PPN

	l.movhi	r27, 0x0007
	l.ori	r27, r27, 0xe000	# r27 = 0x0007e000
	l.and	r27, r27, r23
	l.srli	r27, r27, 13		# r27 = EA[18:13] = set no.

	l.ori	r29, r25, 0x0001	# set valid bit
	l.mtspr	r27, r29, 0x0a00	# ->match[set no.]
	l.ori	r29, r25, 0x03c0	# set all permissions
	l.mtspr	r27, r29, 0x0a80	# ->trans[set no.]

	l.rfe				# return from exception
  \end{lstlisting}
  \caption{The Reset and DTLB Miss exception handlers}
  \label{fig:exh}
\end{figure}

\begin{figure}[t]
  \centering
  \begin{lstlisting}[language={[x86masm]Assembler}]
	# store test value in memory
	l.movhi	r13, 0xc0d1		#
	l.ori	r13, r13, 0xf1ed	# r13 = 0xcod1f1ed
	l.ori	r15, r0, 0x4444		# r15 = 0x4444
	
	l.sw	0(r15), r13

	# activate mmu
	l.mfspr	r13, r0, 0x0011		# r13 <- sr
	l.ori	r13, r13, 0x20		# enable bit 5 (dmmu)
	l.mtspr	r0, r13, 0x0011		# r13 -> sr
	
	# try to load memory location
	l.lwz	r15, 0(r15)
	\end{lstlisting}
  \caption{Main function of first test program}
  \label{fig:mn1}
\end{figure}

\subsection{After Hardware Modification}
The second program was written after the hardware modifications detailed in Section~\ref{sec:har}. In addition to the old exception handlers, which remain unchanged, the Page Fault exception handler now writes a distinctive value to a register and loops indefinitely. In this way it is obvious when a page fault occurs.

The main program stores a test value to memory as before, then sets the appropriate bit in one the new registers to mark the whole cache line as invalid. It then activates the DMMU and attempts to access the memory location, which should trigger a page fault.

The code is given in Figure~\ref{fig:mn2}. As detailed in Section~\ref{sec:har}, the register index is given by bits 18 to 10 of the EA, and the bit index by bits 9 to 5.

\begin{figure}[t]
  \centering
  \begin{lstlisting}[language={[x86masm]Assembler}]
	# store test value in memory
	l.movhi	r13, 0xc0d1		#
	l.ori	r13, r13, 0xf1ed	# r13 = 0xcod1f1ed
	l.ori	r15, r0, 0x4444		# r15 = 0x4444
	
	l.sw	0(r15), r13
	
	# mark cache line as invalid in dmmu
	l.ori	r19, r0, 1
	l.andi	r17, r15, 0x1f		# r17 = bit no = EA[9:5]
	l.sll	r19, r19, r17		# r19 = one-hot(r17)

	l.movhi r17, 0x0007
	l.ori	r17, r17, 0xfb00
	l.and	r17, r15, r17
	l.srli	r17, r17, 10		# r17 = reg no = EA[18:10]

	l.mfspr	r21, r17, 0x0e00
	l.or	r21, r19, r21
	l.sw	4(r15), r21
	l.mtspr	r17, r21, 0x0e00	# set prot reg

	# activate mmu
	l.mfspr	r13, r0, 0x0011		# r13 <- sr
	l.ori	r13, r13, 0x20		# enable bit 5 (dmmu)
	l.mtspr	r0, r13, 0x0011		# r13 -> sr
	
	# try to load memory location
	l.lwz	r15, 0(r15)
	\end{lstlisting}
  \caption{Main function of second test program}
  \label{fig:mn2}
\end{figure}
\chapter{Results}
\label{cha:results}

\section{Hardware Comparison}
\section{Conclusion}
\section{Future Work}

\begin{thebibliography}{99}

\bibitem{heartbleed}
  Heartbleed Website\\
  \url{heartbleed.com}\\
  Fetched 2015-06-15.
  
\bibitem{sockit}
  SoCKit page on Distributor site\\
  \url{parts.arrow.com/item/detail/arrow-development-tools/sockit}\\
  Fetched 2015-06-16.

\bibitem{de2}
	DE2 page on Altera site\\
  \url{wl.altera.com/education/univ/materials/boards/de2/unv-de2-board.html}\\
  Fetched 2015-06-17.
  
\bibitem{fusesoc}
  FuseSoC repository on Github\\
  \url{github.com/olofk/fusesoc}\\
  Fetched 2015-06-16.
  
\bibitem{wb_dec}
	OrpSoC-cores repository on Github\\
	\url{github.com/openrisc/orpsoc-cores/blob/14fb5aa0d96aea7e2594d0bb42195c9477dd33f8/cores/wb_avalon_bridge/verilog/avalon_to_wb_bridge.v} \\
	Fetched 2015-06-16.
	  
\bibitem{wb_inst}
	OrpSoC-cores repository on Github\\
	\url{https://github.com/openrisc/orpsoc-cores/blob/70b775e2ec2e53d15e3f595336e05053ba81f139/systems/sockit/rtl/verilog/orpsoc_top.v#L587} \\
	Fetched 2015-06-16.
	
\bibitem{or1k-linux}
	Linux info on the OpenCores website\\
	\url{opencores.org/or1k/Linux}\\
	Fetched 2015-06-16.
	
\bibitem{or1k-openocd}
	Debugging info on the OpenCores website\\
	\url{opencores.org/or1k/Debugging_physical_targets_(FPGA/ASIC)}\\
	Fetched 2015-06-16.
	
\bibitem{openocd-sockit}
	Discussion of bug with OpenOCD and the SoCKit\\
	\url{comments.gmane.org/gmane.comp.debugging.openocd.devel/26474}\\
	Fetched 2015-06-16.
	
\bibitem{openocd-fix}
	OpenOCD patch on Github\\
	\url{github.com/fjullien/openOCD/commit/0c834568e020ed3b61a055086d9d142f03860e0f}\\
	Fetched 2015-06-16.

\end{thebibliography}

\end{document}
