% !TEX TS-program = pdflatex
% !TEX encoding = UTF-8 Unicode

% This is a simple template for a LaTeX document using the "article" class.
% See "book", "report", "letter" for other types of document.

\documentclass[11pt]{article} % use larger type; default would be 10pt

\usepackage[utf8]{inputenc} % set input encoding (not needed with XeLaTeX)

%%% Examples of Article customizations
% These packages are optional, depending whether you want the features they provide.
% See the LaTeX Companion or other references for full information.

%%% PAGE DIMENSIONS
\usepackage{geometry} % to change the page dimensions
\geometry{a4paper} % or letterpaper (US) or a5paper or....
% \geometry{margin=2in} % for example, change the margins to 2 inches all round
% \geometry{landscape} % set up the page for landscape
%   read geometry.pdf for detailed page layout information

\usepackage{graphicx} % support the \includegraphics command and options

% \usepackage[parfill]{parskip} % Activate to begin paragraphs with an empty line rather than an indent

%%% PACKAGES
\usepackage{booktabs} % for much better looking tables
\usepackage{array} % for better arrays (eg matrices) in maths
\usepackage{paralist} % very flexible & customisable lists (eg. enumerate/itemize, etc.)
\usepackage{verbatim} % adds environment for commenting out blocks of text & for better verbatim
\usepackage{subfig} % make it possible to include more than one captioned figure/table in a single float
% These packages are all incorporated in the memoir class to one degree or another...

%%% HEADERS & FOOTERS
\usepackage{fancyhdr} % This should be set AFTER setting up the page geometry
\pagestyle{fancy} % options: empty , plain , fancy
\renewcommand{\headrulewidth}{0pt} % customise the layout...
\lhead{}\chead{}\rhead{}
\lfoot{}\cfoot{\thepage}\rfoot{}

%%% SECTION TITLE APPEARANCE
\usepackage{sectsty}
\allsectionsfont{\sffamily\mdseries\upshape} % (See the fntguide.pdf for font help)
% (This matches ConTeXt defaults)

%%% ToC (table of contents) APPEARANCE
\usepackage[nottoc,notlof,notlot]{tocbibind} % Put the bibliography in the ToC
\usepackage[titles,subfigure]{tocloft} % Alter the style of the Table of Contents
\renewcommand{\cftsecfont}{\rmfamily\mdseries\upshape}
\renewcommand{\cftsecpagefont}{\rmfamily\mdseries\upshape} % No bold!

\usepackage{hyperref}

%%% END Article customizations

%%% The "real" document content comes below...

\title{Soft Processors}
\author{Jake Humphrey}
%\date{} % Activate to display a given date or no date (if empty),
         % otherwise the current date is printed 

\begin{document}
\maketitle
\tableofcontents
\section{Abstract}
This document seeks to provide an impartial comparison of the various Soft Processors on the market.

\section{Comparison of Soft Processors}
The following metrics will be taken into account:
\begin{itemize}
\item Bus Used
\item Architecture (MIPS/ ARM)
\item Cache
\item MMU (Memory Management Unit)
\item Protection/User Levels
\item License
\item Documentation
\item Toolchain/Debugger/Monitor/JTAG
\item Maturity
\item Resources, Speed on target FPGA
\end{itemize}

\subsection{OpenRISC}
\paragraph{Developer} OpenCores
\paragraph{Bus Used} OpenRISC uses Wishbone, an open source specification by OpenCores.
\paragraph{Architecture} 32/64-bit Mips-like.
\paragraph{Cache} 1--64kB for Instructions and Data each.
\paragraph{MMU} Virtual Memory support. TLB with page size 8kB and 16--256 entries.
\paragraph{User Levels} User/Supervisor Mode.
\paragraph{License} There are two main open source implementations available. A Verilog implementation, OpenRISC 1200, is released under LGPL. Another Verilog implementation, mor1kx, is available under a weak copy-left licence created specifically for it, the OHDL.
\paragraph{Documentation} The specification can be found \href{https://github.com/openrisc/doc/blob/master/openrisc-arch-1.1-rev0.pdf?raw=true}{here}. Documentation for both implementations are also available at their respective Github pages: \href{https://github.com/openrisc/mor1kx}{Mor1kx}, \href{https://github.com/openrisc/or1200}{OR1200}.
\paragraph{Toolchain} A Compiler (based on gcc) and Debugger (based on gdb) are available. See \href{http://opencores.org/or1k/OpenRISC_GNU_tool_chain#Tools}{here}.
\paragraph{Maturity} The OpenRISC 1000 specification has been under development since 2001. Version 1.0 was released in 2012, and Version 1.1 in 2014. The OpenRISC 1200 implementation is stable but not in active development, and is the widely used version according to \href{http://openrisc.github.io/}{the OpenRISC project overview}. Mor1kx is still in active development, but is stated to be more sophisticated.

\subsection{LEON2/3/4}
\paragraph{Developer} Aeroflex Gaisler
\paragraph{Bus Used} AMBA2
\paragraph{Architecture} 32-bit. ISA based on SPARC-V8
\paragraph{Cache} LEON2: Instruction and Data caches. Set-associative with 1--4 sets and 1--64kB/set. LRR or LRU replacement. LEON3/4: 1--4 sets, 1--256 kB/set. Random, LRR or LRU replacement. LEON4 includes an optional L2 cache. 256-bit internal, 1-4 sets, 16kB--8MB.
\paragraph{MMU} SPARC Reference MMU (SRMMU) with configurable TLB
\paragraph{User Levels} The User and Supervisor modes of SPARC.
\paragraph{License} Gaisler state on their website that LEON2 is available from Atmel (AT697 and AT7913), but I found what appears to be the VHDL source code on \href{https://github.com/Galland/LEON2}{Github}. LEON3 is available under GPL as part of the \href{http://www.gaisler.com/index.php/products/ipcores/soclibrary}{GRLIB} library. LEON4 is only available under a commercial license from Gaisler.
\paragraph{Documentation} \href{http://www.gaisler.com/doc/leon3_product_sheet.pdf}{LEON3} and \href{http://www.gaisler.com/doc/LEON4_32-bit_processor_core.pdf}{LEON4} have documentation.
\paragraph{Toolchain} Being SPARC V8 conformant, compilers and kernels for SPARC V8 can be used with LEON processors. Cross compiler \href{http://www.gaisler.com/index.php/products?option=com_content&task=view&id=147}{BCC}. gdb can be used for debugging.
\paragraph{Maturity} 

\subsection{Lm32}
\paragraph{Developer} Lattice
\paragraph{Bus Used} Wishbone
\paragraph{Architecture} 32-bit. ISA \href{http://sourceware.org/cgen/gen-doc/lm32-insn.html}{here}.
\paragraph{Cache} Configurable between no cache, 8kB I-cache, and 8kB I-cache and 8kB D-cache.
\paragraph{MMU} None.
\paragraph{User Levels} No separate modes.
\paragraph{License} LatticeMico32 is licensed under a free (IP) core license.
\paragraph{Documentation} Documentation is available \href{http://www.latticesemi.com/en/Products/DesignSoftwareAndIP/IntellectualProperty/IPCore/IPCores02/LatticeMico32.aspx#_B63AC1B3E7604B0DA4F80789A0B69A31}{here}, but registration is required.
\paragraph{Toolchain} GCC supports the LM32 since version 4.5.0, and Binutils since 2.19. gdb and Newlib are also supported.
\paragraph{Maturity} Introduced in 2006.

\subsection{Microblaze}
\paragraph{Developer} Xilinx
\paragraph{Bus Used} Customisable between CoreConnect PLB, OPB, FSL, LMB, AXI4.
\paragraph{Architecture} In terms of its instruction set architecture, MicroBlaze is very similar to the RISC-based DLX architecture, which in turn is based upon the MIPS architecture.
\paragraph{Cache} 2kB--64kB Direct mapped write-through or write-back.
\paragraph{MMU} 
\paragraph{User Levels} 
\paragraph{License} 
\paragraph{Documentation} 
\paragraph{Toolchain} 
\paragraph{Maturity} 

\subsection{Nios}
\paragraph{Developer} 
\paragraph{Bus Used} 
\paragraph{Architecture} 
\paragraph{Cache} 
\paragraph{MMU} 
\paragraph{User Levels} 
\paragraph{License} 
\paragraph{Documentation} 
\paragraph{Toolchain} 
\paragraph{Maturity} 

\subsection{Nios II}
\paragraph{Developer} 
\paragraph{Bus Used} 
\paragraph{Architecture} 
\paragraph{Cache} 
\paragraph{MMU} 
\paragraph{User Levels} 
\paragraph{License} 
\paragraph{Documentation} 
\paragraph{Toolchain} 
\paragraph{Maturity} 

\subsection{Cortex-M1}
\paragraph{Developer} 
\paragraph{Bus Used} 
\paragraph{Architecture} 
\paragraph{Cache} 
\paragraph{MMU} 
\paragraph{User Levels} 
\paragraph{License} 
\paragraph{Documentation} 
\paragraph{Toolchain} 
\paragraph{Maturity} 

\end{document}
